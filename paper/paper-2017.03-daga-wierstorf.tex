\documentclass[a4paper, 10pt, twocolumn]{article}
% Sowohl LaTeX als auch pdfLaTeX können benutzt werden, um das Manuskript zu erstellen.

% Bitte öffnen sie diese Datei mit utf8 Zeichenkodierung!!!
\usepackage[utf8]{inputenc}         % Schriftkodierung dieser Datei
\usepackage[german]{babel}          % für deutsche Dokumente

\usepackage{graphicx}               % optional für Grafiken
\usepackage{tabularx}               % optional für Tabellen
\usepackage{multirow}               % optional für Tabellen
\usepackage{url}                % optional für Internet Links

\usepackage[small,bf]{caption2}     % bitte für Bildunterschriften verwenden
\usepackage{parskip}
\usepackage{titlesec}
\usepackage{amsmath}                % optional für Formeln

\titleformat{\section}{\normalfont\large\bfseries}{\thesection}{}{}
\titleformat{\subsection}{\normalfont\large\bfseries}{\thesection}{}{}
\titleformat{\paragraph}{\normalfont\bfseries}{\theparagraph}{}{}
\titlespacing{\section}{0pt}{6pt}{-1pt}
\titlespacing{\subsection}{0pt}{3pt}{-1pt}
\titlespacing{\paragraph}{0pt}{3pt}{-1pt}

\newcolumntype{Y}{>{\centering\arraybackslash}X}    %für Tabellen mit tabularx

% Definition der Seitenränder
\addtolength{\textwidth}{2.1cm}
\addtolength{\topmargin}{-2.4cm}
\addtolength{\oddsidemargin}{-1.1 cm}
\addtolength{\textheight}{4.5cm}
\setlength{\columnsep}{0.7cm}

\pagestyle{empty}                   % weder Kopf- noch Fußzeile auf 1. Seite

\begin{document}

\date{}                                         % kein Datum auf 1. Seite

\title{\vspace{-8mm}Wissenschaftliche Erkenntnis, Reproduzierbarkeit und
praktische Lösungen in der Akustik}

% Hier die Namen und Daten der beteiligten Autoren eintragen
\author{
Hagen Wierstorf$^1$, Sascha Spors$^2$, Matthias Geier$^2$\\
$^1$ \emph{\small Filmuniversität Babelsberg KONRAD WOLF, 14482 Potsdam,
Deutschland, Email: hagen.wierstorf@posteo.de
}\\
$^2$ \emph{\small Institut für Nachrichtentechnik, Universität Rostock, 18119 Rostock, Deutschland
}\\
} \maketitle
\thispagestyle{empty}           % weder Kopf- noch Fußzeile auf Folgeseiten
% Beginn des eigentlichen Manuskripts
\section*{Einleitung}
\label{sec:Einleitung}

Reproduzierbarkeit von Ergebnissen stellt eines der wichtigsten Fundamente der
Wissenschaft dar, auch wenn diese das Problem des induktiven Beweises mit sich
bringt und nicht immer leicht quantifizierbar ist (Lewens, 2015). Der Begriff
der Reproduzierbarkeit kann dabei einen weiten Bereich umfassen, angefangen von
bloßer Replikation von Ergebnissen eines Algorithmus (Drummond, 2009) bis hin zu
der Etablierung der Ergebnisse durch deren erfolgreiche Verwendung in vielen
anderen wissenschaftlichen Publikationen (Charlton, 2007). Die stetige Zunahme
wissenschaftlicher Publikationen bei gleichzeitig steigender Abhängigkeit der
Ergebnisse von großen Datenmengen, statistischer Analyse, numerischer Simulation
und umkämpften Fördergeldern führen unweigerlich zu der Frage, ob dies
Auswirkungen auf die Reproduzierbarkeit hat. Zumal einige medizinische Studien
ein erschreckend geringes Maß an Reproduzierbarkeit gezeigt haben (z.B.
Ioannidis, 2005). Dieser Beitrag geht der Frage nach wie ein solches Problem in
der Akustik verhindert werden kann. Dazu stellt es ein exemplarisches Projekt
aus der Schnittstelle zwischen der virtuellen Akustik und Psychoakustik vor und
zeigt wie mit unterschiedlichen Tools (R, python, Jupyter, SOFA, zenodo) die
Reproduzierbarkeit sowohl auf Ebene der Replikation als auch im Hinblick auf den
gemeinsamen wissenschaftlichen Fortschritt in der Akustik verbessert werden
kann.

\textbf{Outline}
\begin{itemize}
    \item Was ist wissenschaftliche Methodik
    \item Typen der Reproduzierbarkeit wieder aufnehmen?
    \item Unterschied zwischen Reproduzierbarkeit derselben
        Ergebnisse/Abbildungen und Reproduzierbarkeit im Sinne der
        Verallgemeinerung (Induktion) erklären
    \item Darauf eingehen, dass statistische Signifikanz kein Beweis ist?
    \item Beispiel WFS mixing Versuch (Vortrag von Chris auf DAGA):
        \begin{itemize}
            \item SFS Python Numerik Plot [Python, Jupyter]
            \item (SFS Matlab Skript zum Erzeugen der Stimuli) [SOFA, zenodo]
            \item Python+R Skript zur statischen Auswertung [Python, R,
                Jupyter?]
        \end{itemize}
\end{itemize}

Seit der DAGA 2015 können die Manuskripte bis zu vier Seiten enthalten.


\begin{thebibliography}{5}
\bibitem{ArticleReference}
Schall, A.: How to write a manuscript. Acta Acustica united with
Acustica 90 (2004), 2203-2503
\bibitem{BookReference}
Klang, B.: Akustik im Überblick. Schall und Rauch Verlag, Stadt,
2010
\bibitem{URLReference}
DAGA 2017 Homepage, URL:\\
\url{http://www.daga2017.de/}
\bibitem{PDFCreator}
PDFCreator, URL:\\
\url{http://sourceforge.net/projects/pdfcreator}
\bibitem{Ghostware}
Free software Ghostview and Ghostscript, URL:\\
\url{http://www.cs.wisc.edu/~ghost/}
\end{thebibliography}
\end{document}
