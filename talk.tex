\documentclass{beamer}

\usetheme{avad}

\usepackage[utf8]{inputenc}
\usepackage{amsmath,amssymb}

%\graphicspath{%
%{05_psychoacoustics/}%
%}



%%%%%%%%%%%%%%%%%%%%%%%%%%%%%%%%%%%%%%%%%%%%%%%%%%%%%%%%%%%%%%%%%%%%%%%%%%%%%%%%
%%%%%%%%%%%%%%%%%%%%%%%%%%%%%%%%%%%%%%%%%%%%%%%%%%%%%%%%%%%%%%%%%%%%%%%%%%%%%%%%
\begin{document}

%%%%%%%%%%%%%%%%%%%%%%%%%%%%%%%%%%%%%%%%%%%%%%%%%%%%%%%%%%%%%%%%%%%%%%%%%%%%%%%%
\title{Wissenschaftliche Erkenntnis, Reproduzierbarkeit und praktische Lösungen
in der Akustik}
\subtitle{Hagen Wierstorf$\;^1$\\Sascha Spors$\,^2$\\Matthias Geier$\,^2$\\%
          \vspace{0.5cm}%
          \footnotesize{%
          $^1\,$Filmuniversität Bablesberg \emph{KONRAD WOLF}\\%
          $^2\,$Institut für Nachrichtentechnik, Universität Rostock}}
\date{07.03.2017}
\maketitle

%%%%%%%%%%%%%%%%%%%%%%%%%%%%%%%%%%%%%%%%%%%%%%%%%%%%%%%%%%%%%%%%%%%%%%%%%%%%%%%%
\begin{frame}{Motivation}

\end{frame}

%%%%%%%%%%%%%%%%%%%%%%%%%%%%%%%%%%%%%%%%%%%%%%%%%%%%%%%%%%%%%%%%%%%%%%%%%%%%%%%%
\begin{frame}{Wissenschaftliche Methode}

    \begin{itemize}
        \item \textbf{Deduktiv:} \\
            \small{Dachs ist ein Säugetier\\%
                   Dieter ist ein Dachs\\%
                   $\Rightarrow$ Dieter ist ein Säugetier}
        \item \textbf{Induktiv:} \\
            \small{Medikament hatte keine Nebenwirkung an $100\,000$ getesteten
                   Menschen.\\%
                   Kann sicher verwendet werden.}
        \item \textbf{Computer Simulation:} \dots
    \end{itemize}

\end{frame}

%%%%%%%%%%%%%%%%%%%%%%%%%%%%%%%%%%%%%%%%%%%%%%%%%%%%%%%%%%%%%%%%%%%%%%%%%%%%%%%%
\begin{frame}{Wann ist etwas Wissenschaft?}

    Überprüfung einer Aussage durch Falsifizierbarkeit
    \begin{quote}
        If it disagrees with experiment, it's wrong. In that simple statement is
        the key to science.
    \end{quote}

    {\ft\hfill R. Feynman zitiert nach T. Lewens (2015)}

    \vspace{1cm}

    \begin{itemize}
        \item Nicht automatisch Widerlegung einer Theorie (Gran Sasso)
        \item Kann als Abgrenzung von Pseudowissenschaften verwendet werden
            (Astronomie)
        \item Reproduzierbarkeit von Ergebnissen wichtig
    \end{itemize}

\end{frame}

%%%%%%%%%%%%%%%%%%%%%%%%%%%%%%%%%%%%%%%%%%%%%%%%%%%%%%%%%%%%%%%%%%%%%%%%%%%%%%%%
\begin{frame}{Gründe für Nicht-Reproduzierbarkeit}

\end{frame}

%%%%%%%%%%%%%%%%%%%%%%%%%%%%%%%%%%%%%%%%%%%%%%%%%%%%%%%%%%%%%%%%%%%%%%%%%%%%%%%%
\begin{frame}{Statistik}

\end{frame}

%%%%%%%%%%%%%%%%%%%%%%%%%%%%%%%%%%%%%%%%%%%%%%%%%%%%%%%%%%%%%%%%%%%%%%%%%%%%%%%%
\begin{frame}{Auswege}

\end{frame}

%%%%%%%%%%%%%%%%%%%%%%%%%%%%%%%%%%%%%%%%%%%%%%%%%%%%%%%%%%%%%%%%%%%%%%%%%%%%%%%%
\begin{frame}{Neue Publikationswege}

    Dies hat technische und prinzipielle Komponenten.

\end{frame}

%%%%%%%%%%%%%%%%%%%%%%%%%%%%%%%%%%%%%%%%%%%%%%%%%%%%%%%%%%%%%%%%%%%%%%%%%%%%%%%%
\begin{frame}{Bibliography}

    \ft
    \begin{thebibliography}{36}
         \bibitem[10]{Lindosland2011}
            Lindosland, Wikimedia Commons, Public Domain, 2011.
        \bibitem[11]{Mills1958}
            Mills, A. W. (1958). On the minimum audible angle. J. Acoust. Soc.
            Am., 30(4), 237--46.
        \bibitem[12]{Blauert1997}
            Blauert, J. (1997). Spatial Hearing. The MIT Press.
        \bibitem[13]{Mills1972}
            Mills, A. W. (1972). Auditory Localization. In Tobias, J. V. (Ed.),
            Foundations of Modern Auditory Theory (Vol. II), Academic Press.
    \end{thebibliography}
           
\end{frame}

\end{document}
